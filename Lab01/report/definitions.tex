\chapter{Определения}
% \section*{Определение $\gamma$-доверительного интервала}

%\section*{Равномерное распределение}
\textbf{Определение}. Говорят, что случайная величина $X$ имеет равномерное распределение на отрезке $[a; b]$, если ее функция плотности имеет вид

\begin{equation*}
	f_{X}(x) =
	\begin{cases}
		c, & x \in [a;b]\\
		0, & \text{иначе}
	\end{cases}
\end{equation*}

Значение константы $c$ однозначно определяется из условия нормировки.

\begin{equation*}
	1 = \int\limits_{-\infty}^{+\infty}f(x)dx =
	\int\limits_{a}^{b}cdx =
	c(b - a) = 1
	\Longrightarrow
	c = \frac{1}{b - a}	
\end{equation*}

Обозначается $X \sim R(a, b)$.

Функция распределения равномерной случайной величины X:

\begin{equation*}
	F_X(x) = 
	\begin{cases}
		0, & x < a\\
		\frac{x-a}{b-a}, & x \in [a;b]\\
		1, & x > b
	\end{cases}
\end{equation*}

\newpage
%\section*{Гамма-распределение}
\textbf{Определение} Гамма-функцией Эйлера называется отбражение

\begin{equation*}
	\Gamma: \mathds{R}^{+} \longrightarrow \mathds{R},
\end{equation*}

определенное правилом

\begin{equation*}
	\Gamma(x) =
	\int\limits_0^{+\infty} e^{-t} \cdot t^{x-1} dt
\end{equation*}

\textbf{Определение} Говорят что случайная величина $\xi$ имеет гамма-распределение с параметрами $\lambda$ и $\alpha$, если ее функция плотности распределения вероятностей имеет вид

\begin{equation*}
	f_{\xi}(x) =
	\begin{cases}
		\frac{\lambda^{\alpha}}{\Gamma(\alpha)}x^{\alpha-1}e^{-\lambda x}, & x > 0\\
		0, & \text{иначе}
	\end{cases}
\end{equation*}

Обозначается $\xi \sim \Gamma(\lambda, \alpha)$.