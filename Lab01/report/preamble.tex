% LaTeX settings for MATLAB code listings
% based on Ted Pavlic's settings in http://links.tedpavlic.com/ascii/homework_new_tex.ascii
\usepackage{listings}
\usepackage[usenames,dvipsnames]{color}

% This is the color used for MATLAB comments below
\definecolor{MyDarkGreen}{rgb}{0.0,0.4,0.0}

% For faster processing, load Matlab syntax for listings
\lstloadlanguages{Matlab}%
\lstset{language=Matlab,                        % Use MATLAB
	frame=single,                           % Single frame around code
	basicstyle=\small\ttfamily,             % Use small true type font
	keywordstyle=[1]\color{Blue}\bfseries,        % MATLAB functions bold and blue
	keywordstyle=[2]\color{Purple},         % MATLAB function arguments purple
	keywordstyle=[3]\color{Blue}\underbar,  % User functions underlined and blue
	identifierstyle=,                       % Nothing special about identifiers
	% Comments small dark green courier
	commentstyle=\usefont{T1}{pcr}{m}{sl}\color{MyDarkGreen}\small,
	stringstyle=\color{Purple},             % Strings are purple
	showstringspaces=false,                 % Don't put marks in string spaces
	tabsize=5,                              % 5 spaces per tab
	%
	%%% Put standard MATLAB functions not included in the default
	%%% language here
	morekeywords={xlim,ylim,var,alpha,factorial,poissrnd,normpdf,normcdf},
	%
	%%% Put MATLAB function parameters here
	morekeywords=[2]{on, off, interp},
	%
	%%% Put user defined functions here
	morekeywords=[3]{FindESS, homework_example},
	%
	morecomment=[l][\color{Blue}]{...},     % Line continuation (...) like blue comment
	numbers=left,                           % Line numbers on left
	firstnumber=1,                          % Line numbers start with line 1
	numberstyle=\tiny\color{Blue},          % Line numbers are blue
	stepnumber=1,                           % Line numbers go in steps of 5
	breaklines=true,
	extendedchars=\true,
	keepspaces=true
}

% Includes a MATLAB script.
% The first parameter is the label, which also is the name of the script
%   without the .m.
% The second parameter is the optional caption.
\newcommand{\matlabscript}[2]
{\lstinputlisting[caption=#2,label=#1]{#1.m}}

\usepackage[T2A]{fontenc}
\usepackage[utf8]{inputenc}
\usepackage[english,russian]{babel}

\usepackage[left=30mm, right=20mm, top=20mm, bottom=20mm]{geometry}

\usepackage{microtype}
\sloppy

\usepackage{setspace}
\onehalfspacing

\usepackage{indentfirst}
\setlength{\parindent}{12.5mm}

\usepackage{titlesec}
\titleformat{\chapter}{\LARGE\bfseries}{\thechapter}{20pt}{\LARGE\bfseries}
\titlespacing*{\chapter}{\parindent}{*2}{*2}
\titleformat{\section}{\Large\bfseries}{\thesection}{20pt}{\Large\bfseries}

\addto{\captionsrussian}{\renewcommand*{\contentsname}{Содержание}}
\usepackage{natbib}
\renewcommand{\bibsection}{\chapter*{Список использованных источников}}

\usepackage{caption}

\usepackage{wrapfig}
\usepackage{float}

\usepackage{graphicx}
\newcommand{\imgwc}[4]
{
	\begin{figure}[#1]
		\center{\includegraphics[width=#2]{inc/img/#3}}
		\caption{#4}
		\label{img:#3}
	\end{figure}
}
\newcommand{\imghc}[4]
{
	\begin{figure}[#1]
		\center{\includegraphics[height=#2]{inc/img/#3}}
		\caption{#4}
		\label{img:#3}
	\end{figure}
}
\newcommand{\imgsc}[4]
{
	\begin{figure}[#1]
		\center{\includegraphics[scale=#2]{inc/img/#3}}
		\caption{#4}
		\label{img:#3}
	\end{figure}
}

\usepackage{pgfplots}
\pgfplotsset{compat=newest}

\newcommand{\code}[1]{\texttt{#1}}

\usepackage{amsmath}
\usepackage{amssymb}

\usepackage[unicode]{hyperref}
\hypersetup{hidelinks}

\makeatletter
\newcommand{\vhrulefill}[1]
{
	\leavevmode\leaders\hrule\@height#1\hfill \kern\z@
}
\makeatother

%added
\usepackage{ dsfont }