\chapter{Текст программы}
В листингах \ref{Generator.cs}--\ref{EventModel.cs} представлен код программы, отвечающий за моделирование.

\begin{lstinputlisting}[
	caption={Код генератора},
	label={Generator.cs},
	language={[Sharp]C}
	]{inc/src/Generator.cs}
\end{lstinputlisting}

\newpage
\begin{lstinputlisting}[
	caption={Код заявки},
	label={Request.cs},
	language={[Sharp]C}
	]{inc/src/Request.cs}
\end{lstinputlisting}

\begin{lstinputlisting}[
	caption={Код очереди заявок},
	label={ReqQue.cs},
	language={[Sharp]C}
	]{inc/src/ReqQue.cs}
\end{lstinputlisting}

\newpage

\begin{lstinputlisting}[
	caption={Код обслуживающего аппарата},
	label={Service.cs},
	language={[Sharp]C}
	]{inc/src/Service.cs}
\end{lstinputlisting}


\newpage
\begin{lstinputlisting}[
	caption={Код модели $\Delta$t},
	label={DeltaTModel.cs},
	language={[Sharp]C}
	]{inc/src/DeltaTModel.cs}
\end{lstinputlisting}

\newpage
\begin{lstinputlisting}[
	caption={Код событийной модели},
	label={EventModel.cs},
	language={[Sharp]C}
	]{inc/src/EventModel.cs}
\end{lstinputlisting}