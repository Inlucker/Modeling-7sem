\chapter{Теория}
\section{Принцип $\Delta t$}
Принцип $\Delta t$ заключается в последовательном анализе состояний всех блоков в момент $t + \Delta t$ по заданному состоянию блоков в момент t. При этом новое состояние блоков определяется в соответствии с их алгоритмическим описанием с учетом действующих случайных факторов, задаваемых распределениями вероятности. В результате (такого анализа)принимается решение о том, какие общесистемные события должны имитироваться программной моделью на данный момент времени.

Основной \textbf{недостаток} этого принципа: значительные затраты машинного времени на реализацию моделирования системы. А при недостаточно малом $\Delta t$ появляется опасность пропуска отдельных событий в системе, что исключает возможность получения адекватных результатов при моделировании. \textbf{Достоинство}: равномерная протяжка времени.

\section{Событийный принцип}
Характерное свойство моделируемых систем обработки информации то, что состояние отдельных устройств изменяются в дискретные моменты времени, совпадающие с моментами времени поступления сообщений в систему, временем поступления окончания задачи, времени поступления аварийных сигналов и т.д. Следовательно моделирование и продвижение времени в системе удобно проводить, используя событийный принцип. При использовании данного принципа, состояние всех блоков имитационной модели анализируется лишь в момент появления какого-либо события. Момент поступления следующего события определяется минимальным значением из списка будущих событий, представляющего собой совокупность моментов ближайшего изменения состояния каждого из блоков системы.

\textbf{Недостаток} событийного принципа: (самостоятельная обработка)

\section{Программа}
Условием остановки поиска является обслуживание 100 000 сообщений без изменения максимальной длины очереди.   

В случае, если такое событие не происходит за 1 000 000 заявок, принимается, что генерация вместе с обратной связью помещают сообщения с большей интенсивностью, чем успевает обрабатывать их ОА. Следовательно, со временем длина очереди будет в среднем только расти, поэтому для любой выбранной очереди в определенный момент произойдут потери.