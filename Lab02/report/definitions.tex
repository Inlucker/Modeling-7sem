\chapter{Отчет}
\section{Теория}
Для математического описания функционирования устройств, развивающегося в форме случайного процесса, может быть с успехом применен математический аппарат, разработанный в теории вероятности для так называемых Марковских случайных процессов.

Случайный процесс протекающий в некоторой системе S, называется Марковским, если он обладает следующим свойством: для каждого момента времени, вероятность любого состояния системы в будущем зависит только от ее состояния в настоящем и не зависит от того, когда и каким образом система пришла в это состояние.

Для Марковского процесса обычно составляются уравнения Колмогорова, представляющие следующие соотношения:

$F = (p'(t), p(t), \Lambda)=0$, где $\lambda$ - набор параметров.

Интегрирование системы дает искомые вероятности состояний, как функций от времени. Начальные условия берутся в зависимости от того, какого было начальное состояние системы.

\section{Программа}
Условием стабилизации вероятности $i$-ого состояния принимается величина $P_i(t)$, где $t$ - наименьшее время, при котором ${P'}_i(t) < 10^{-8}$. 

Также реализована возможность задавать 2 начальных условия:
\begin{itemize}
	\item в нулевой момент времени система находится в первом состоянии;
	\item в нулевой момент времени система находится в каждом состоянии с равной вероятностью;
\end{itemize}